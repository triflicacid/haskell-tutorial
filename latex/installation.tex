\section{Installation}
\label{sec:installation}

Link: \url{https://www.haskell.org/ghcup/}

I used GHCup to install several components of the Haskell toolchain.

\subsection{The Haskell Toolchain}
The Haskell Toolchain consists of several useful tools for Haskell compilatio and development:

\begin{itemize}
  \item \textbf{GHC} - the Glasgow Haskell Compiler;
  \item \textbf{cabal-install} - Cabal installation tool for managing Haskell software;
  \item \textbf{Stack} - a cross-platform proram for developing Haskell projects;
  \begin{itemize}
    \item \textbf{Msys2} - provides a UNIX shell and environment which is necessary for executing configuration scripts.
  \end{itemize}
  \item \textbf{haskell-language-server} - a language server which may be integrated into an IDE;
\end{itemize}

\subsection{Command}
The command to use on Windows (in a normal Powershell instance) is

\input{latex/latex-listings-powershell.tex}

\begin{lstlisting}[language=PowerShell]
  Set-ExecutionPolicy Bypass -Scope Process -Force;[System.Net.ServicePointManager]::SecurityProtocol = [System.Net.ServicePointManager]::SecurityProtocol -bor 3072; try { Invoke-Command -ScriptBlock ([ScriptBlock]::Create((Invoke-WebRequest https://www.haskell.org/ghcup/sh/bootstrap-haskell.ps1 -UseBasicParsing))) -ArgumentList $true } catch { Write-Error $_ }
\end{lstlisting}