\section{Type Classes}
This section covers more complex type classes. For basics, see \texttt{Types -> Type Classes}.

\subsection{Semigroups}
A type is a Semigroup if there exists some function which, when two values from the group are combined, that value is also in the Semigroup.

\begin{lstlisting}[language=haskell]
class Semigroup a where
  (<>) :: a -> a -> a

infixr 6 <>
\end{lstlisting}

This diamond operator must be \textbf{associative}:
\begin{lstlisting}[language=haskell]
x <> (y <> z) == (x <> y) <> z
\end{lstlisting}

\subsection{Monoid}
A Monoid is an extension of a Semigroup, adding an identity element.
\begin{lstlisting}[language=haskell]
class Semigroup a => Monoid a where
  mempty :: a

  -- Optional functions
  mappend :: a -> a -> a
  mappend = (<>)

  mconcat :: [a] -> a
  mconcat = foldr mappend mempty
\end{lstlisting}

\texttt{mempty} must be an identity element for the Semigroup. Therefore, it must obey:
\begin{lstlisting}[language=haskell]
x <> mempty == x
mempty <> x == x
\end{lstlisting}

\subsubsection{Multiple Monoids}
A type may have multiple Monoid implementations (i.e. multiple viable operators exist which satisfy Monoidal conditions).

For example, take numerical types:
\begin{lstlisting}[language=haskell]
newtype Sum n = Sum { getSum :: n }
instance Num a => Semigroup (Sum a) where
  (<>) = (+)
instance Num a => Monoid (Sum a) where
  mempty = 0


newtype Product n = Product { getProduct :: n }
instance Num a => Semigroup (Product a) where
  (<>) = (*)
instance Num a => Monoid (Product a) where
  mempty = 1

\end{lstlisting}

\subsection{Foldable}
A Foldable datatype is one which may be reduced to a single value i.e. folded in on itself.
\begin{lstlisting}[language=haskell]
class Foldable t where
  foldr :: (a -> b -> b) -> b -> t a -> b
\end{lstlisting}

\subsection{Functors}
A Functor is a type class which may have an operation mapped over it.

\begin{lstlisting}[language=haskell]
class Functor f where
  fmap :: (a -> b) -> f a -> f b -- Infix symbol is <$>

  -- The following functions are optional
  (<$) :: a -> f b -> f a
  (<$) = fmap . const
\end{lstlisting}

A Functor instance must obery the following laws:
\begin{itemize}
  \item The mapping must preserver the structure of the arguments.
  \item \textbf{Identity}: \begin{lstlisting}[language=haskell]
fmap id == id\end{lstlisting}
  \item \textbf{Distributive over Composition}: \begin{lstlisting}[language=haskell]
fmap (f . g) == fmap f . fmap g\end{lstlisting}
\end{itemize}

\subsubsection{Fmap}
\texttt{fmap} allows a function to be mapped over a structure without the internal structure of the Functor changing.

\textbf{Example}:
\begin{lstlisting}[language=haskell]
data Tree a = Leaf a | Node (Tree a) a (Tree a)

instance Functor Tree where
  fmap f (Leaf a) = Leaf (f a)
  fmap f (Node a b c) = Node (fmap f a) (f b) (fmap f c)
\end{lstlisting}

\subsection{Applicatives}
Applicatives are Functors with more and better functionality, with \texttt{<*>} essentially \textit{injecting} a value into a wrapped function, and \texttt{pure} allowing easy construction of an applicative.

\begin{lstlisting}[language=haskell]
class Functor f => Applicative f where
  pure :: a -> f a
  (<*>) :: f (a -> b) -> f a -> f b

  -- Optional functions
  (*>) :: f a -> f b -> f b
  a *> b = b
  (<*) :: f a -> f b -> f a
  a <* b = a
\end{lstlisting}

An Applicative must obey the following laws:
\begin{itemize}
  \item \textbf{Identity}: \begin{lstlisting}[language=haskell]
pure id <*> v == v\end{lstlisting}
  \item \textbf{Homomorphism}: \begin{lstlisting}[language=haskell]
pure f <*> pure x == pure (f x)\end{lstlisting}
  \item \textbf{Interchange}: \begin{lstlisting}[language=haskell]
u <*> pure y == pure (\$ y) <*> u\end{lstlisting}
  \item \textbf{Composition}: \begin{lstlisting}[language=haskell]
pure (.) <*> u <*> v <*> w == u <*> (v <*> w)\end{lstlisting}
\end{itemize}

\subsection{Monads}
A Monad allows the transformation of a value into a Monad via a function.

\begin{lstlisting}[language=haskell]
class Applicative m => Monad m where
  (>>=) :: m a -> (a -> m b) -> m b -- "Bind"

  -- Optional functions
  (>>) :: m a -> m b -> m b -- "Then"
  a >> b = a >>= \x -> b

  return :: a -> m a
  return = pure
\end{lstlisting}

Any implementation must abide by these laws:
\begin{itemize}
  \item \textbf{Left identity}: \begin{lstlisting}[language=haskell]
return a >>= h == h a\end{lstlisting}
  \item \textbf{Right identity}: \begin{lstlisting}[language=haskell]
m >>= return == m\end{lstlisting}
  \item \textbf{Associativity}: \begin{lstlisting}[language=haskell]
(m >>= g) >>= h == m >>= (\x -> g x >>= h)\end{lstlisting}
\end{itemize}

\subsubsection{Bind}
\bordertext{\texttt{(>>=) :: \red{Monad} m => m a -> (a -> m b) -> m b}}
This function takes a monad and a function which takes a raw value and returns a new monad, and returns another new monad.

When implemented, then, we may vary the action taken depending on the value of the provided monad, such as returning a default value -- this is what \texttt{(>>=)} does with \texttt{Maybe}, as shown below: 

\textbf{Example}:
\begin{lstlisting}[language=haskell]
  add :: Num a => Maybe a -> Maybe a -> Maybe a
  add mx my = mx >>= (\x -> my >>= (\y -> Just (x + y)))

  -- Then addition works as expected
  add (Just 1) (Just 2) -- => Just 3
  -- And if either one of the arguments is Nothing, it returns Nothing
  add Nothing (Just ?) -- => Nothing
\end{lstlisting}

\subsubsection{Then}
\bordertext{\texttt{(>>) :: \red{Monad} m => m a -> m b -> m a}}

This function discards the second monad given to it. \texttt{m >> n} is equivalent to \texttt{m >>= \symbol{92}\_ -> n}.

``Then'' can be though of wanting to carry out an action but not caring what the result is.

\subsubsection{Return}
\bordertext{\texttt{return :: \red{Monad} m => a -> m a}}

Return wraps a monad around a raw value.

Using the example from the Bind section, we could substitute the explicit \texttt{Just} with the more general \texttt{return}. Now, this would theoretically work with any appropriately-defined monad.

\begin{lstlisting}[language=haskell]
  add mx my = mx >>= (\x -> my >>= (\y -> return (x + y)))
\end{lstlisting}

\subsubsection{Fail}
\bordertext{\texttt{fail :: \red{Monad} m => \red{String} -> m a}}

Fail is intended to be called when something goes wrong. The default implementation is to call \texttt{error} (i.e. error out of the program), but it may be implemented so that certain errors may be handled and return an appropriate monad as a response.

\subsubsection{``do'' Syntax}

Chaining together applications of \texttt{(>>=)}, \texttt{(>>)} and lambda functions can get tedious; that's where the syntactic sugar ``do'' expression comes in.

The ``statements'' inside of \texttt{do} are executed in order, and if one ``statement'' fails this will be propagated through.

\begin{itemize}
  \item \textbf{Bind} \begin{lstlisting}[language=haskell]
    m >>= \x -> ...
    -- Becomes
    do
      x <- m
      ...
  \end{lstlisting}
  \item \textbf{Then} \begin{lstlisting}[language=haskell]
    m >> ...
    -- Becomes
    do
      m
      ...
  \end{lstlisting}
\end{itemize}

\textbf{Example}:
\begin{lstlisting}[language=haskell]
  -- Re-writing the above definition of 'add'
  add mx my = do
    x <- mx
    y <- my
    return $ x + y
\end{lstlisting}

\subsubsection{Kleisli-Composition}
This operator, denoted \texttt{>=>}, acts as a Monad composition operator. It is defined in \texttt{Control.Monad} as so
\begin{lstlisting}[language=haskell]
infixr 1 >=>
(>=>) :: Monad m => (a -> m b) -> (b -> m c) -> a -> m c
f >=> g = \x -> f x >>= g
\end{lstlisting}
