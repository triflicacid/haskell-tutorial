\section{Functions}

Functions are defined to map a value from an input set - the \textit{domain} - into an output set - \textit{the co-domain}. Every output of the function is contained in a subset of the co-domain, called the \textit{range}. \textbf{Pure} mathematical functions must be able to map every value from the domain, and each input value must map to only one output value.

In Haskell, occurences of functions are expanded into their RHS.

\subsection{Function Definition}

Functions are defined by providing its name, a list of arguments, and setting it equal to an expression.

\begin{center}
  \texttt{\textcolor{blue}{name} arg1 arg2 \ldots argn = <expr>}
\end{center}

The arguments may be set to constants, or may be given a name to accept a variable value.

The type of a function is specified by an arrow (\texttt{->}) seperated list of its argument types and its return type:

\begin{center}
  \texttt{\blue{func} :: \red{type$_1$} -> \red{type$_2$} -> \ldots -> \red{type$_n$} -> \red{type$_\text{return}$}}
\end{center}

For an example, take a function which returns the sum of the elements of an array:

\begin{lstlisting}[language=haskell]
  sum :: [a] -> Int -- Takes an array of arbitrary type and returns an integer
  sum [] = 0 -- Define the sum of an empty array to be zero
  sum (h:t) = h + sum t -- Define the sum of an array to be the head plus the sum of the tail
\end{lstlisting}

\subsubsection{Infix Functions}

A good example of infix functions are operators such as \texttt{+}. Functions which take two arguments may be writen between the arguments instead.

For example, say we had \texttt{\blue{add} a b = a + b}. Then \texttt{\blue{add} 5 7} and \texttt{5 \;\textasciigrave\blue{add}\textasciigrave\; 7} are equivalent.

To reference the function defined by an operator i.e. \texttt{+}, surround it with parenthesis i.e. \texttt{(+)}.

The associativity and precedence of infix operators may be changed via \texttt{\blue{infixr}/\blue{infixl} <prec> <function name>} where $0 < \texttt{prec} \le 9$.

\subsubsection{Pattern Matching}

Parameters of functions may be matched against a pattern. Note that order matters: more specific patterns should be defined first, with general cases last.
\begin{itemize}
  \item Accept any value by using a symbol e.g. \texttt{f x = ...}.
  \item Accept a certain value e.g. \texttt{f 0 = ...}, \texttt{g [] = ...}
  \item List splicing using \texttt{(x:xs)} where \texttt{x} is the head, \texttt{xs} is the tail.
  \item Tuple unpacking e.g. \texttt{(Int,Int)} could be unpacked using \texttt{(x,y)}.
  \item Accept and extract a custom types. E.g., say we had \texttt{\blue{type} Pos = (Int, Int)}, we would define \texttt{getX (x, y) = x}.
  \item Accept and extract a custom datatype. E.g., say we had \texttt{\blue{data} Num = Zero | Succ Num}. We could then define \texttt{f Zero = ...} and \texttt{f (Succ n) = ...}.
\end{itemize}

Pattern matching may also be done using \texttt{\blue{case} ... \blue{of} ...} construct.

\subsection{Function Application}

A function is applied (called) to some arguments as follows:

\begin{center}
  \texttt{\blue{name} arg1 arg2 \ldots argn}
\end{center}

For example, consider the function \texttt{\blue{in\_range} x min max = x >= min \&\& x < max}, an implementation of $x \in [min, max)$.

Then, \texttt{\blue{in\_range} 3 0 5} would evaluate to True, but \texttt{\blue{in\_range} 5 0 5} would evaluate to False.

\subsection{Recursion}
Recursion is the process of a function calling itself. Recursion requires a \textit{base case} to stop the function recursing indefinitely.

There are many ways to implement recursion, which will be demonstrated using the \textit{factorial}, defined as
\[n! = n \cdot (n - 1) \cdot \ldots \cdot 1 = \prod_{k = 1}^{n} k\]

\subsubsection{Defined Base Case}
We can hard-code the case where the function is called with the base case:

\begin{lstlisting}[language=haskell]
  fac 1 = 1
  fac n = n * fib (n-1) 
\end{lstlisting}

\subsubsection{If-Else Expression}
We can use the if-else expression:

\begin{center}
  \texttt{\blue{if} <expr> \blue{then} <ifTrue> \blue{else} <ifFalse>}
\end{center}

For example,

\begin{lstlisting}[language=haskell]
  fac n = if n <= 1 then 1 else n * fac (n-1)
\end{lstlisting}

\subsubsection{Guards}
Guards are similar to piece-wise functions.

\begin{lstlisting}[language=haskell]
<main_expr>
  | <expr> = <value>
  | <expr2> = <value2>
  ...
  | otherwise = <default_value>
\end{lstlisting}
Where \texttt{<expr>} is a boolean expression. If \texttt{<expr>} is matches, then \texttt{<value>} will be returned. If none is matched, the \texttt{\blue{otherwise}} is returned.

For example,

\begin{lstlisting}[language=haskell]
  fac n =
    | n <= 1    = 1
    | otherwise = n * fac (n-1)
\end{lstlisting}

\subsubsection{Accumulators}
In this example, we define an auxiliary function \texttt{\blue{aux}} inside \texttt{\blue{fac}} to calculate the the factorial

\begin{lstlisting}[language=haskell]
  fac n = aux n 1
    where
      aux n acc
        | n <= 1    = acc
        | otherwise = aux (n-1) (n*acc)
\end{lstlisting}

This is called \textit{tail recursion}. This is because the final result of \texttt{\blue{aux}} is the result we want, meaning that it is much more memory efficient. A good compiler could even unwind this into a non-recursive imperative approach using a loop. (For more insight, see \url{https://www.youtube.com/watch?v=_JtPhF8MshA}.)

Normal recursion (using an above definition of \texttt{\blue{fac}}):
\begin{verbatim}
fac 4
= 4 * fac 3
= 4 * (3 * fac 2)
= 4 * (3 * (2 * fac 1))
= 4 * (3 * (2 * 1))
= 4 * (3 * 2)
= 4 * 6
= 24
\end{verbatim}

Tail recursion (using the definition in this sub-section):
\begin{verbatim}
fac 4
= aux 3 4
= aux 2 12
= aux 1 24
= 24
\end{verbatim}

\subsection{Lambdas}
Syntax:
\begin{center}
  \texttt{(\blue{\symbol{92}}<args> -> <expr>)}
\end{center}

Some examples:
\begin{itemize}
  \item \texttt{(\symbol{92}x -> x+1) 2} returns \texttt{3}.
  \item \texttt{(\symbol{92}x y z -> x+y+z) 1 2 3} returns \texttt{6}.
\end{itemize}
Lambdas may be bound to names.

\subsection{Higher Order Functions}
Higher order functions are functions that take other functions as arguments.

For example, a function which takes another function and applies it to an argument:
\begin{lstlisting}[language=haskell]
  app :: (a -> b) -> a -> b
  app f x = f x
\end{lstlisting}
A synonym of such a function is the dollar (\texttt{\$}) operator: \texttt{(\$) :: (a -> b) -> a -> b}.

\subsubsection{Useful Higher Order Functions}
\begin{itemize}
  \item \texttt{\blue{map} :: (a -> b) -> [a] -> [b]} applies a function to every element on an array.
  \[L' = \{f(x) : x \in L\}\]
  
  Example: \texttt{\blue{map} (\symbol{92}x -> x\string^2) [1,2,3]} returns \texttt{[1,4,9]}.
  \item \texttt{\blue{filter} :: (a -> \red{Bool}) -> [a] -> [a]} filters the list on a predicate.
  \[L' = \{x \in L : P(x)\}\]
  
  Example: \texttt{\blue{filter} (\symbol{92}x -> mod x 2 == 0) [1,2,3,4,5]} returns \texttt{[2,4]}.
  \item \texttt{\blue{fold} :: (a -> b -> b) -> b -> [a] -> b} processes a list with some function to produce a single value, starting at a given value.
  
  In Haskell, \texttt{\blue{fold}} doesn't exist, but rather \texttt{\blue{foldr}} and \texttt{\blue{foldl}} wherein the position of the base value is different

  \begin{lstlisting}[language=haskell]
foldr f a [] = a
foldr f a (x:xs) = f x (foldr f a xs)
-- foldr f a [x1, x2, ..., xn] = (x1 `f` (x2 `f` (... (xn `f` a))))

foldl f a = []
foldl f a (x:xs) = f (foldl f a xs) x
-- foldl f a [x1, x2, ..., xn] = (...((a `f` x1) `f` x2) `f` xn)
  \end{lstlisting}

  See \texttt{code/folding.hs}

  Example: \texttt{\blue{foldr} (+) 0 [1,2,3,4,5]} returns \texttt{15}.
\end{itemize}

\subsection{Currying}
The principle behind currying is that given
\begin{center}
  \texttt{\blue{f} :: a -> b -> c -> d}
\end{center}
We could re-write this as
\begin{center}
  \texttt{\blue{f'} :: a -> (b -> (c -> d))}
\end{center}

For example, one could define a function \texttt{\blue{add}} in multiple ways:
\begin{lstlisting}[language=haskell]
  add x y = x + y
  add x = (\y -> x + y)
  add = (\x -> (\y -> x + y))
\end{lstlisting}

\subsubsection{Currying}
\begin{lstlisting}[language=haskell]
curry :: ((a,b -> c) -> a -> b -> c)
curry f x y = f (x,y)
\end{lstlisting}

\subsubsection{Uncurrying}
\begin{lstlisting}[language=haskell]
uncurry :: (a -> b -> c) -> (a,b) -> c
uncurry f (x,y) = f x y
\end{lstlisting}

\subsubsection{Partial Function Application}
Using the last definition of \texttt{\blue{add}}, consider the result of \texttt{\blue{add} 1}. This would be a new function; \texttt{\blue{add} 1 :: \red{Int} -> \red{Int}}. This is known as a \textit{section}.

A good example would be using \texttt{\blue{map}}.
\begin{center}
  \texttt{\blue{doubleList} = \blue{map} (\symbol{92}x -> x * 2)}
\end{center}

\subsection{Function Composition}
Function composition is a way to combine functions. For this, we use the dot (\texttt{.}) operator.
\begin{center}
  \texttt{(.) :: (b -> c) -> (a -> b) -> (a -> c)}
\end{center}
Then, \texttt{(f.g)} is equivalent to \texttt{(\blue{\symbol{92}}x -> f (g x))}.

For example, all three definitions of \texttt{\blue{descSort}} are equivalent:
\begin{lstlisting}[language=haskell]
  descSort = reverse . sort
  descSort = (\x -> reverse (sort x))
  descSort = reverse (sort x)
\end{lstlisting}
