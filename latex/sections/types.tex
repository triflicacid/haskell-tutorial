\section{Types}

Every expression in Haskell has a type. Types are inferred, even when given explicitly.

Types always begin with an UPPERCASE letter.

The cardinality of a type is how many \textit{states} the data type could hold. Polymorphic types have no cardinality.

\subsection{Variable Types}

\begin{center}
  \texttt{var :: \red{type}}
\end{center}

\subsubsection{Lists}

To define a list of a \texttt{\red{type}}, one would write \texttt{[\red{type}]}. This may be nested.

\subsection{Function Types}

\begin{center}
  \texttt{func :: \red{type1} -> ... -> \red{typeN} -> \red{ret\_type}}
\end{center}

Where the function \texttt{func} takes $n$ arguments of types \texttt{\red{type1}, ..., \red{typeN}} and returns \texttt{\red{ret\_type}}.

\subsubsection{Type Variables}

Type variables may be used where any type would be permissable and must be lowercase. For example,

\begin{lstlisting}[language=haskell]
  id :: a -> a
  id x = x
\end{lstlisting}

This is called ``parametric polymorphism''.

Contraints may be imposed on type variables; this is known as ``ad-hoc polymorphism''.
\begin{lstlisting}[language=haskell]
  abs :: Num a => a -> a
  abs x = if x < 0 then -x else x
\end{lstlisting}
Multiple type constraints may be listed by comma-seperating them inside parentheses.

\subsection{Type Aliasing}
This doesn't define a new datatype, but rather an alias for another type.
\begin{center}
  \texttt{\blue{type} Pos = \red{type}}
\end{center}

\begin{lstlisting}[language=haskell]
  type Pos = (Int, Int)
  getX :: Pos -> Int
  getX (x,y) = x

  -- May have type parameters
  type Dual a = Either a a
\end{lstlisting}

\subsection{Type Classes}
Type classes may be used to restrict the types a polymorphic function may take. This is useful if we would like to use features in a polymorphic function that may only be available to certain types. For a type to be a member of a type class, it must implement all of the required methods.

To impose a constraint on variable \texttt{a} in a function \texttt{f}: \texttt{f :: (\texttt<TypeClass> a, ...) => ...}.

\subsubsection{Definition}
A type class definition has the syntax
\begin{lstlisting}[language=haskell]
class <Name> <var> where
  -- Function signatures
  ...
\end{lstlisting}
For a type to be a member, it must implements every function listed. A class definition may include constraints.

\subsubsection{Implementation}
To define a type to be an instance of a type class:
\begin{lstlisting}[language=haskell]
instance <TypeClass> <TypeName> where
  -- Function definitions
  ...
\end{lstlisting}

\textbf{Example}:
\begin{lstlisting}[language=haskell]
data S = S String deriving Show

class Show a => Exclaim a where
    exclaim :: a -> a

instance Exclaim S where
    exclaim = (++ "!") . show
    
main = putStrLn $ exclaim (S "hello")
\end{lstlisting}

\subsubsection{Common Type Classes}
Common type classes include:
\begin{itemize}
  \item \texttt{\red{Eq}} -- types which may be compared i.e. \texttt{(==)} is defined;
  \item \texttt{\red{Num}} -- numeric types, gives us access to standard mathematical operations i.e. \texttt{(+), (-), (*), abs, ...} are defined;
  \item \texttt{\red{Ord}} -- types which may be ordered, imposes a total ordering i.e. \texttt{(<), (>), (<=)} are defined;
  \item \texttt{\red{Read}} -- types which may be converted from a string i.e. \texttt{read} is defined;
  \item \texttt{\red{Show}} -- types which may be converted to a string i.e. \texttt{show} is defined;
  \item \texttt{\red{Integral}} -- types which are integer-like i.e. \texttt{div, ...} is defined;
  \item \texttt{\red{Floating}} -- types which are float-like i.e. \texttt{(/), ...} is defined;
  \item \texttt{\red{Enum}} -- types which may be enumerated i.e. \texttt{succ, pred, ...} are defined;
\end{itemize}

\textit{N.B. for information on complex type classes, see Type Class section}

\subsection{Defining Datatypes}
The \blue{data} keyword is used to define a new datatype; unlike the above, these are entirely custom.

This may be done using the \texttt{data} keyword:
\begin{center}
  \texttt{\blue{data} Name = \red{Constructor1} [<args>] | \ldots}
\end{center}
where \texttt{<args>} are the \textit{types} of each argument, not literals.

Constructors are either plain values, or functions which take \texttt{args} and return the datatype.

Data constructors can include polymorphism by including type variables after \texttt{<Name>} e.g. \texttt{\blue{data} Maybe a = Nothing | Just a}.

\subsubsection{Examples}
\begin{itemize}
  \item \url{rock-paper-scissors.hs} -- A basic example revolving around Rock-Paper-Scissors;
  \item \url{expr.hs} -- A program to build and evaluate expressions;
  \item \url{tree.hs} -- A representation of a tree structure;
  \item \url{nat-num.hs} -- A definition of natural numbers using the successor function;
\end{itemize}

\subsubsection{Derivation}
The \texttt{\blue{deriving}} keyword can be used to automatically generate implementations for the given type class(es).

Syntax: \texttt{\blue{data} <Name> = \ldots \blue{deriving} (<Class1>, \ldots)}

\textbf{Example}:
\begin{lstlisting}[language=haskell]
data Shape = Circle Int | Rect Int Int
  deriving (Show)
  
print (Circle 5) = "Circle 5
\end{lstlisting}

Only certain classes may be derived, such as \texttt{Show}, \texttt{Read}, \texttt{Ord} etc.
Language extensions \texttt{DeriveFunctor}, \texttt{DeriveFoldable} and \texttt{DeriveTraversable} may be enabled to allow derivation of Functor, Foldable and Traversable classes.

\textbf{Standalone Deriving}
The language extension \texttt{StandaloneDeriving} allows datatypes to derive type classes seperate from its definition. This allows one to derive instances even when one doesn't have access to the data definition.
\begin{lstlisting}[language=haskell]
{-# LANGUAGE StandaloneDeriving #-}

data Foo a = Bar a | Baz String
-- ...
deriving instance Show a => Show (Foo a)
\end{lstlisting}

\subsubsection{Generalised Algebraic Datatypes}
GADTs add pattern matching and constants to data definitions, and allows data constructors to have more complex return types.

\begin{lstlisting}[language=haskell]
data <Name> <TypeVariable> where
  <Variant> :: <Constructor> <Type>
\end{lstlisting}

For example,
\begin{lstlisting}
{-# LANGUAGE GADTs #-}

-- Without GADTs, every rteurn type would be 'Term a'
data Term a where
  Lit    :: Int -> Term Int
  Succ   :: Term Int -> Term Int
  IsZero :: Term Int -> Term Bool
  If     :: Term Bool -> Term a -> Term a -> Term a
  Pair   :: Term a -> Term b -> Term (a,b)

eval :: Term a -> a
eval (Lit i)      = i
eval (Succ t)     = 1 + eval t
eval (IsZero t)   = eval t == 0
eval (If b e1 e2) = if eval b then eval e1 else eval e2
eval (Pair e1 e2) = (eval e1, eval e2)
\end{lstlisting}

\subsection{Records}
Records allow data to be stored with an associated name.

Syntax:
\begin{center}
  \texttt{\blue{data} <Name> = <Name> \{ <field> :: <type>, \ldots \}}
\end{center}
This will automatically generate functions \texttt{<field> :: <Name> -> <type>} to extract said properties. This has the side-effect that field names must be globally unique.

\subsubsection{Multiple Constructors}
Note that records may also have mutliple constructors,
\begin{lstlisting}[language=haskell]
  data Point = D2 { x :: Int, y :: Int }
    | D3 { x :: Int, y :: Int, z :: Int }
\end{lstlisting}

Duplicate field names in this context is OK.

This will generate functions \texttt{x}, \texttt{y} and \texttt{z} all with the signature \texttt{x/y/z :: Point -> \red{Int}}. Both \texttt{x} and \texttt{y} will work on either \texttt{D2} or \texttt{D3}, but applying \texttt{z} to \texttt{D2} will throw an exception.

For an example, see \url{code/vector.hs}.