\section{Modules}

In Haskell, each file is a module. Hence, each file (other than the entry file) must begin with a module declaration:

\begin{center}
  \texttt{\blue{module} \textit{filename} [(n1, n2, ...)] \blue{where}}
\end{center}

By default, every symbol is exported. If \texttt{(n1, n2, ...)} is included, only symbols \texttt{n1}, \texttt{n2} etc. are exported.

\subsection{Importing}

To import a module, use an import statement:

\begin{center}
  \texttt{\blue{import} \textit{module} [(n1, n2, ...)]}
\end{center}

By default, every symbol exported by \texttt{module} is imported. If \texttt{(n1, n2, ...)} is included, only symbols \texttt{n1}, \texttt{n2} etc. are imported.

To import \url{Animals.hs} one would write \texttt{\blue{import} Animals}.
To import \url{Farm/Tractor.hs} one would write \texttt{\blue{import} Farm.Tractor}.

Once imported, symbols may be used freely. For example, if the function \texttt{double} is imported, to reference it we would write \texttt{double}. However, if two different definitions for \texttt{double} are imported, we must use its full name e.g. \texttt{Module.double}.

\subsubsection{Qualified Imports}

Syntax:

\begin{center}
  \texttt{\blue{import qualified} \textit{module} [(...)]}
\end{center}

This forces the module name to precede any symbols imported. In the example above, \texttt{Module.double} \textbf{must} be used.

\subsubsection{Aliased Imports}

Syntax:

\begin{center}
  \texttt{\blue{import} \textit{module} [(...)] \blue{as} \textit{alias}}
\end{center}

Using the above example, now, instead of writing \texttt{Module.double} one would now write \texttt{Alias.double}.

\subsubsection{Hiding Imports}
The \texttt{\blue{hiding}} keyword can be used to omit imports.
E.g. \texttt{\blue{import} Prelude \texttt{hiding} (map)} would import every function from \texttt{Prelude} but omit \texttt{map}.

\subsection{Importing Datatypes}
Let's say \texttt{\blue{data} DataType = A $\mid$ B $\mid$ C} is defined inside module \texttt{Module}.

\begin{itemize}
  \item To import every constructor, we'd write \texttt{\blue{import} Module (DataType(..))}
  \item To import only constructor \texttt{A}, we'd write \texttt{\blue{import} Module (DataType(A))}
  \item To import constructors \texttt{A} and \texttt{B}, we'd write \texttt{\blue{import} Module (DataType(A,B))}
\end{itemize}