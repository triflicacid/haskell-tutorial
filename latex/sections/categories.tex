\section{Category Theory}
This section will contain a brief look into category theory and how it applies to Haskell. Various concepts such as Applicatives and Monads are discussed mathematically; for a more haskell-focused approach, see chapter \texttt{Type Classes}.

A category is a collection of \textit{objects} and \textit{arrows}. Arrows acts as pathways between objects.
\begin{itemize}
  \item $A \to B$ is a \textit{morphism} from $A$ to $B$.
  \item $A \to A$ is an \textit{identity morphism} from $A$ to $B$ (endomorphism).
  \item Let $A \to B$ and $B \to C$ be denoted $f$ and $g$, respectively. Then $A \to C$ is denoted as $g \circ f$ as a \textit{composition}.
\end{itemize}
In a category, every object has at least one identity morphism, and every morphism is composable.

Notation:
\begin{itemize}
  \item \texttt{obj(C)} := Class of objects in a category.
  \item \texttt{hom(C)} := Class of morphisms in a category.
  \item \texttt{C(a,b)} := All morphisms from $a$ to $b$.
  \item $\circ$ := Composition of morphisms
  \begin{itemize}
    \item $h \circ f \circ g \equiv (h \circ f) \circ g \equiv h \circ (f \circ g)$
    \item $f \circ 1 \equiv 1 \circ f \equiv f$
  \end{itemize}
\end{itemize}

In Haskell, types can be viewed as categories with functions acting as morphisms. Indeed, every function is composable, and the function \texttt{id} acts as an identity morphism.

\begin{lstlisting}[language=haskell]
import Control.Category

class Category (cat :: k -> k -> *) where
  id :: cat a a
  (.) :: cat b c -> cat a b -> cat a c
\end{lstlisting}

\subsection{Functors}
A functor maps one category to another - in haskell, it maps some computation into the functorial context using \texttt{fmap}.

\begin{lstlisting}[language=haskell]
class Functor (f :: * -> *) where
  fmap :: (a -> b) -> f a -> f b
\end{lstlisting}

The function \texttt{fmap} maps the morphism \texttt{a -> b} to \texttt{f a -> f b}.

\subsection{Monoidal Category}
Given a category $C$, a functor $\diamond$ (the ``tensor product'') where $\diamond : C \times C \to C$, and an identity element $I$ with
\begin{itemize}
  \item $\alpha_{A,B,C} := (A \diamond B) \diamond C \equiv A \diamond (B \diamond C)$
  \item $\lambda_A := I \diamond A \equiv A$
  \item $\rho_A := A \diamond I \equiv A$
\end{itemize}
A monoidal category is given by $(S, \{1\}, \diamond)$ where $S$ is a set, $1$ is the identity element, and $\diamond$ is a tensor product operation.

\subsection{Monoidal Functors}
Given two monoidal ctegories, $(C, 1_C, \diamond_C)$ and $(D, 1_D, \diamond_D)$ then $F : C \to D$ is a monoidal functor with
\begin{itemize}
  \item $\phi_{A,B} := F(A) \diamond_D F(B) = F(A \diamond_C B)$
  \item $\phi := 1_D \to F(1_C)$
\end{itemize}

Let's define a monoidal functor in Haskell.
\begin{lstlisting}[language=haskell]
class Functor f => Monoidal f where
  unit :: f ()
  (**) :: f a -> f b -> f (a, b)
\end{lstlisting}

This monoidal functor is different from a normal functor as \texttt{(**)} only works on objects which are in the same functorial context.

let's further define a function
\begin{lstlisting}[language=haskell]
(<**>) :: Monoidal f => f (a -> b) -> f a -> f b
mf <**> mx = fmap (\(f,x) -> f x) (mf ** mx)

-- With this operator, we can lift any function into a functor
lift2 :: (a -> b -> c) -> (f a -> f b -> f c)
lift2 f x = (<**>) (fmap f x)

lift3 :: (a -> b -> c -> d) -> (f a -> f b -> f c -> f d)
lift3 f a b c = lift2 f a (b <**> c)

...

lift<n> f x1 ... xn = lift<n-1> f x1 ... x<n-1> <**> xn
\end{lstlisting}

\subsection{Applicative Functors}
Applicatives are simply equivalent to lax monoidal functors. These allow functions to be lifted into the functorial context in order to compose them.

\begin{lstlisting}[language=haskell]
class Functor f => Applicative (f :: * -> *) where
  pure :: a -> f a
  (<*>) :: f (a -> b) -> f a -> f b
  liftA2 :: (a -> b -> c) -> f a -> f b -> f c
\end{lstlisting}

\subsection{Monoids}
Given a monoidal category $(C, 1, \diamond)$, then $(M, \mu, \eta)$ is a monoid iff
\begin{itemize}
  \item $M$ is an element in $obj\;(C)$
  \item $\mu : M \diamond M \to M$
  \item $\eta : 1 \to M$
\end{itemize}

\subsection{Monads}
A monad makes it possible to lift a value into the context and access with a function without loosing the context (important!).

Given a catgory $C$ and a functor $T$ with
\begin{itemize}
  \item $T : C \to C$ (endofunctor)
  \item $\eta : 1_C \to T$
  
  Haskell: \texttt{$\eta$ :: $1_C$ -> T (a -> m a)}
  \item $\mu : T^2 \to T$

  Haskell: \texttt{$\mu$ :: $T^2$ -> T (m (m a) -> m a)}
\end{itemize}
Such that
\begin{itemize}
  \item $\mu \circ T \mu \equiv T \mu \circ \mu$
  \item $\mu \circ T \eta \equiv \mu \circ \eta T$
\end{itemize}

By these definitions, a useful property is that any number of applications of $T$ can be reduces into a single application via $\mu$.

A monad takes a category and puts it into a functorial context.

The following snippet illustrates $\eta$ and $\mu$ definitions for \texttt{Maybe} as \texttt{unit} and \texttt{join} respectively,
\begin{lstlisting}[language=haskell]
unit :: a -> Maybe a
unit = Just

join :: Maybe (Maybe a) -> Maybe a
join (Just x) = x
join Nothing = Nothing
\end{lstlisting}

So far we have no functions to work with the values inside the Monad context.
\begin{lstlisting}[language=haskell]
map :: Monad m => (a -> b) -> m a -> m b
map = fmap
\end{lstlisting}
We notice that the signature strongly resembles that of \texttt{fmap} for functors. Indeed, a sufficient definition is simply to re-use \texttt{fmap}.

Combined with the property of applicative functors to not leave their context, it makes sense in Haskell to define a Monad as a subclass of Applicative.
\begin{lstlisting}[language=haskell]
class Applicative m => Monad (m :: * -> *) where
  (>>=) :: m a -> (a -> m b) -> m b
  (>>) :: m a -> m b -> m b
  return :: a -> m a
\end{lstlisting}
using the functions already defined, we can define these new functions
\begin{lstlisting}[language=haskell]
x >>= f = join (map f x)
a >> b = a >>= \_ -> b
return = unit
\end{lstlisting}

\subsection{Arrows}
Arrows is a structure representing the abstract concept of computation, spefically composition, parameterised by their input and output. Arrows are essentially functions lifted into a context.

\begin{lstlisting}[language=haskell]
import Control.Arrow -- Useful functions/classes found here!

class Category a => Arrow (a :: * -> * -> *) where
  arr :: (b -> c) -> a b c

  -- optional functions
  first :: a b c -> a (b, d) (c, d)
  second :: a b c -> a (d, b) (d, c)
  (***) :: a b c -> a b' c' -> a (b, b') (c, c')
  (&&&) :: a b c -> a b c' -> a b (c, c')
\end{lstlisting}

\begin{itemize}
  \item \texttt{arr} lifts a function into the arrow context. It takes a function \texttt{input -> output} and returns an Arrow instance with the same input and output; can be through of like \texttt{pure} for Arrows.
  \item \texttt{first} takes an existing arrow, and creates a new arrow which works on tuples. The function operates on the first argument of the tuple and preserves the second.
  \item \texttt{second} takes an existing arrow, and creates a new arrow which works on tuples. The function operates on the second argument of the tuple and preserves the first.
  \item \texttt{(***)} is a combination of both \texttt{first} and \texttt{second}, returning tuples which contain both the origin and transformed inputs.
  \item \texttt{(\&\&\&)} transforms an argument in two different ways, returning both outputs.
\end{itemize}

\subsubsection{Function Type}
\begin{lstlisting}[language=haskell]
instance Arrow (->) where
  -- These are both necessary
  arr = id
  (***) f g (x, y) = (f x, f y)

  -- These are optional
  first f (x, y) = (f x, y)
  second f (x, y) = (x, f y)
  (&&&) f g (x, y) = (f x, g y)
\end{lstlisting}

\subsubsection{Kleisli Arrows}
\begin{lstlisting}[language=haskell]
newtype Kleisli m a b = Kleisli { runKleisli :: a -> m b }

instance Monad m => Arrow (Kleisli m) where
  arr f = Kleisli (return . f)
  first (Kleisli f) = Kleisli (\ ~(b,d) -> f b >>= \c -> return (c,d))
  second (Kleisli f) = Kleisli (\ ~(d,b) -> f b >>= \c -> return (d,c))
\end{lstlisting}

\subsubsection{Choice Arrows}
\begin{lstlisting}[language=haskell]
class Arrow a => ArrowChoice (a :: * -> * -> *) where
  left :: a b c -> a (Either b d) (Either c d) -- Only changes value of the Left constructor
  right :: a b c -> a (Either d b) (Either d c) -- Only changes value of the Right constructor
  (+++) :: a b c -> a b' c' -> a (Either b b') (Either c c')
  (|||) :: a b c -> a c d -> a (Either b c) d
\end{lstlisting}

\subsubsection{Arrow Application}
\begin{lstlisting}[language=haskell]
class Arrow a => ArrowApply (a : * -> * -> *) where
  app :: a (a b c, b) c

instance ArrowApply (->) where
  app (f, x) = f x
\end{lstlisting}