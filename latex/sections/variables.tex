\section{Variables}

Variables are name-bounded values. Variables in Haskell are immutable - they cannot be changed.

\begin{center}
  \texttt{\blue{name} = <value>}
\end{center}

Variable types are \textit{inferred}. To explicitly assign variables a type,

\begin{center}
  \texttt{\blue{name} :: \red{Type}}
\end{center}

In GHCi, the type of a symbol may be retrieved by \texttt{:t \blue{name}}.

\subsection{Local Name Binding}
Two methods are provided to bind a symbol to a value in a local scope: \texttt{let} and \texttt{where}.

\subsubsection{Let}
Syntax:
\begin{lstlisting}[language=haskell]
  let
    <symbol> = <expr>
    <symbol2> = <expr2>
    ...
  in
  <main_expr>
\end{lstlisting}

For example, re-define \texttt{\blue{in\_range}} as follows:

\begin{lstlisting}[language=haskell]
  in_range x min max =
    let
      in_lb = min <= x
      in_ub = max > x
    in
    in_lb && in_ub
\end{lstlisting}

\subsubsection{Where}
Syntax:

\begin{lstlisting}[language=haskell]
  <main_expr>
  where
    <symbol> = <expr>
    <symbol2> = <expr2>
    ...
\end{lstlisting}

For example, re-define \texttt{\blue{in\_range}} as follows:

\begin{lstlisting}[language=haskell]
  in_range x min max = in_lb && in_ub
    where
      in_lb = min <= x
      in_ub = max > x
\end{lstlisting}