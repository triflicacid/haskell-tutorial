\section{Type Families}

Type families are functions at defined at the type level.

\textbf{Example}:
\begin{lstlisting}[language=haskell]
data Nat = Z | S Nat

-- value-level functions
add :: Nat -> Nat -> Nat
add Z b = b
add (S a) b = S (add a b)

-- Type-level type family
type family Add (a :: Nat) (b :: Nat) :: Nat where
  Add 'Z b = b
  Add ('S a) b = 'S (Add a b)
\end{lstlisting}

In the first line of the family definition, we list the types of the arguments and the rteurn type. We populate the block with function definitions.
Generally, every equation defined up-front: these are called \textbf{closed} type families. We can also have \textbf{open} type families which do not define every equation.

\begin{lstlisting}[language=haskell]
data Bool = True | False

-- Type-level equivalent to match the value-level function
type family Not (b :: Bool) :: Bool where
  Not 'True = 'False
  Not 'False = 'True

not :: Bool b -> Bool (Not b)
not True = False
not False = True
\end{lstlisting}

The input type determines what the output type will be. The type family will be evaluated for each pattern match in our function equation.