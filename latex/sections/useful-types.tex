\section{Useful Types}
This section will list some common, useful types which should be known.

\subsection{Maybe}
The \texttt{Maybe} type is incredibly useful, as it can be used to represent the \textit{absence} of a value. This is useful when our function is passed invalid data, for example.

If it defined as: \texttt{\blue{data} Maybe a = Nothing | Just a}

\paragraph{Functions}
Found inside \texttt{Data.Maybe}.
\begin{itemize}
  \item \texttt{isJust :: Maybe a -> \red{Bool}} -- returns if the passed \texttt{Maybe} is a \texttt{Just} value;
  \item \texttt{fromMaybe :: a -> Maybe a -> a} -- returns the \texttt{Just} value if a value is present, else returns the default value;
  \item \texttt{fromJust :: Maybe a -> a} -- returns the \texttt{Just} value, or throws an exception if recieved \texttt{Nothing};
  \item \texttt{catMaybes :: [Maybe a] -> [a]} -- returns a list containing all the \texttt{Just} values.
  \item \texttt{mapMaybe :: (a -> Maybe b) -> [a] -> [b]} -- maps a function over a list of \texttt{Maybe}s, discarding any \texttt{Nothing}s;
  \item \texttt{maybe :: b -> (a -> b) -> Maybe a -> b} -- takes a maybe. If \texttt{Just}, applies a function and returns. Else, returns a default;
\end{itemize}

\paragraph{Instances}
\begin{lstlisting}[language=haskell]
instance Functor Maybe where
  fmap _ Nothing = Nothing
  fmap f (Just a) = Just (f a)
instance Applicative Maybe where
  pure = Just
  Nothing <*> _ = Nothing
  Just f <*> a = fmap f a
instance Monad Maybe where
  (Just a) >>= f = f a
  Nothing >>= _ = Nothing
\end{lstlisting}

\subsection{Either}
The \texttt{Either} type can be used to represent a union of types -- a value which is either one type, or another.

It is defined as \texttt{\blue{data} Either a b = Left a $\mid$ Right b}

\paragraph{Functions}
Found inside \texttt{Data.Either}.
\begin{itemize}
  \item \texttt{lefts :: [Either a b] -> [a]} -- returns an array of left-hand side values;
  \item \texttt{rights :: [Either a b] -> [b]} -- returns an array of right-hand side values;
  \item \texttt{isLeft :: Either a b -> \red{Bool}} -- returns whether the provided value is a left-hand side value;
  \item \texttt{isRight :: Either a b -> \red{Bool}} -- returns whether the provided value is a right-hand side value;
  \item \texttt{fromLeft :: a -> Either a b -> a} -- returns the left-hand side value if a \texttt{Left} is provided, else returns a default;
  \item \texttt{fromRight :: n -> Either a b -> n} -- returns the right-hand side value if a \texttt{Right} is provided, else returns a default;
  \item \texttt{either :: (a -> c) -> (b -> c) -> Either a b -> c} -- processes the left- or right-hand value in the \texttt{Either} as per the given functions;
  \item \texttt{partitionEithers :: [Either a b] -> ([a],[b])} -- traverses the list, placing and \texttt{Left} values in one list and any \texttt{Right} values in another;
\end{itemize}

\paragraph{Instances}
\begin{lstlisting}[language=haskell]
instance Functor Either where
  fmap _ (Left a) = Left a
  fmap f (Right a) = Right (f a)
instance Applicative Either where
  pure = Right
  Left f <*> _ = Left f
  Right f <*> a = fmap f a
instance Monad Either where
  Left a >>= _ = Left a
  Right a >>= f = f a
\end{lstlisting}

\paragraph{Use -- Error Handling}
\texttt{Either} can be used in place of \texttt{Maybe} for error handling.

\texttt{Left} can represent failiure with the error type attached, and \texttt{Right} representing success with the result attached.

\subsection{State}
State is used to contain a computational context which can be preserved inside of it. It is defined as:
\begin{lstlisting}[language=haskell]
newtype State s a = State { runState :: s -> (s, a) }

-- Constructor has the following signature:
(s -> (s, a)) -> State s a
\end{lstlisting}

\paragraph{Instances}

\begin{lstlisting}[language=haskell]
instance Functor (State s) where
  fmap = liftA

instance Applicative (State s) where
  pure x = State (\s -> (s, x))
  ff <*> fa = do { f <- ff; a <- fa; pure (f a) }

instance Monad (State s) where
  mx >>= f = State $ \s -> let (s', x) = runState mx s
    in runState (f x) s'

  ma >> mb = State $ \s -> let (s', _) = runState ma s
    in runState mb s'
\end{lstlisting}

\subsection{Environment Functor}

As \texttt{(->)} is an operator at the type-level, we can partially apply it. E.g., \texttt{(->) a} is a function which accepts type \texttt{a}.

\begin{lstlisting}[language=haskell]
instance Functor ((->) r) where
  fmap = (.)
  -- f <$> g = \ r -> f (g r)

instance Applicative ((->) r) where
  f <*> g = \ r -> f r (g r)

instance Monad ((->) r) where
  g >>= f = \ r -> f (g r) r
\end{lstlisting}