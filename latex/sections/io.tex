\section{I/O}

I/O produces an issue with Haskell as I/O functions aren't \textit{pure}.

\subsection{The \texttt{IO} Type}
All I/O functions in Haskell have the following type: \texttt{IO \red{<value>}}.

This special type holds a given I/O action. When the \texttt{IO} value is used, the stored action will be carried out, and \texttt{IO \red{<value>}} is returned.

For example, in GHCI

\texttt{> hi = putStrLn "Hello, World!"}

\texttt{> hw}

\texttt{Hello, World!}

Notice how nothing was outputted until the \texttt{IO} value was used. Note that \texttt{hw} may be used mutliple times.

\subsection{Input}
\begin{itemize}
  \item \texttt{getLine :: IO \red{String}} -- retrieves a line of input from STDIN;
\end{itemize}

\subsection{Output}
\begin{itemize}
  \item \texttt{putStr :: \red{String} -> IO ()} -- puts the given string to STDOUT;
  \item \texttt{putStrLn :: \red{String} -> IO ()} -- puts the given string to STDOUT on a new line;
\end{itemize}

\subsection{Extracting \texttt{<value>}}
\texttt{IO} is a \textit{monad}, and should be extracted as such.

\texttt{greet :: IO ()}

\texttt{greet = \blue{do}}

\quad\texttt{putStrLn "What's your name? "}

\quad\texttt{name <- getLine}

\quad\texttt{putStrLn \$ "Hello, " ++ name ++ "!"}

You can only extract values from \texttt{IO} inside of another \texttt{IO} action.

For a more complex example, see \url{code/IO.hs}.