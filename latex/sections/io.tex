\section{I/O}

I/O produces an issue with Haskell as I/O functions aren't \textit{pure}.

\subsection{The \texttt{IO} Type}
All I/O functions in Haskell have the following type: \texttt{IO \red{<value>}}.

This special type holds a given I/O action. When the \texttt{IO} value is used, the stored action will be carried out, and \texttt{IO \red{<value>}} is returned.

For example, in GHCI

\texttt{> hi = putStrLn "Hello, World!"}

\texttt{> hw}

\texttt{Hello, World!}

Notice how nothing was outputted until the \texttt{IO} value was used. Note that \texttt{hw} may be used mutliple times.

\subsection{Input}
\begin{itemize}
  \item \texttt{getLine :: IO \red{String}} -- retrieves a line of input from STDIN;
  \item \texttt{readLn :: Read a => IO a} -- retrieves a line of input from STDIN, reading it as specified by \texttt{a};
\end{itemize}

\subsection{Output}
\begin{itemize}
  \item \texttt{putStr :: \red{String} -> IO ()} -- puts the given string to STDOUT;
  \item \texttt{putStrLn :: \red{String} -> IO ()} -- puts the given string to STDOUT on a new line;
  \item \texttt{print :: Show a => a -> IO ()} -- essentially the same as \texttt{putStrLn . show};
\end{itemize}

\subsection{Extracting \texttt{<value>}}
\texttt{IO} is a \textit{monad}, and should be extracted as such.
\begin{lstlisting}[language=haskell]
greet :: IO ()
greet = do
  putStrLn "What is your name?"
  name <- getLine
  putStrLn $ "Hello, " ++ name ++ "!"
\end{lstlisting}

You can only extract values from \texttt{IO} inside of another \texttt{IO} action.

For a more complex example, see \url{code/IO.hs}.

\subsection{Environment}
The following functions are defined in \texttt{System.Environment}.

\subsubsection{Command-Line Arguments}
Command-line arguments are arguments passed to the executable e.g. \texttt{./prog.exe arg1 arg2 ...}

These can be accessed via \texttt{getArgs :: IO [\red{String}]}

Note, the program name ``\texttt{prog.exe}'' is ommited; this can be access via \texttt{getProgName :: IO String}

\subsubsection{Environment Variables}
The function \texttt{getEnvironment :: IO [(String, String)]} gets a list of all environment variables in name-value pairs.

To get only one variable, the function \texttt{lookupEnv :: String -> IO (Maybe String)} returns the value of an environment variable.

The function \texttt{withArgs :: [String] -> IO a -> IO a} loads sets the environment variables inside an IO action.

\subsubsection{Error Handling}
Defined in \texttt{System.Exit}

\begin{itemize}
  \item \texttt{exitWith :: ExitCode -> IO a} exits the program with the provided exit code (\texttt{ExitCode = ExitSuccess | ExitFailure Int});
  \item \texttt{exitSuccess :: IO a} exits the program with exit code of success;
  \item \texttt{exitFailure :: IO a} exits the program with exit code of failure (1);
  \item \texttt{die :: String -> IO a} prints the given message, then exits the program with exit code of failure (1);
\end{itemize}

\subsection{Files}
The symbols \texttt{stdout :: Handle} and \texttt{stdin :: Handle} are handles to the process' input/output streams. I/O functions defined above use the following functions with these handles provided.

\begin{itemize}
  \item \texttt{readFile :: FilePath -> IO \red{String}} -- reads contents of the file
  \item \texttt{writeFile :: FilePath -> \red{String} -> IO ()} -- writes to the file (overwrites contents if exists)
  \item \texttt{appendFile :: FilePath -> \red{String} -> IO ()} -- appends to the file
  \item \texttt{renameFile :: FilePath -> FilePath -> IO ()} -- renamed the given file to the second argument
  \item \texttt{deleteFile :: \red{String} -> IO ()} -- deletes the given file
\end{itemize}

Alternatively, you can use file handles.
\begin{itemize}
  \item \texttt{openFile :: FilePath -> IOMode -> IO Handle} -- Open a file in the given mode.
  
  \texttt{\blue{data} IOMode = ReadMode | WriteMode | AppendMode | ReadWriteMode}
  \item \texttt{hGetContents :: Handle -> IO String} -- Get contents of the file
  \item \texttt{hPutStr :: Handle -> \red{String} -> IO ()} -- writes the given string to the file
  \item \texttt{hPrint :: Show a => Handle -> a -> IO ()} -- converts \texttt{a} to a string and write to the file
  \item \texttt{hClose :: Handle -> IO ()} -- Close the given handle
  \item \texttt{withFile :: FilePath -> IOMode -> (Handle -> IO a) -> IO a} -- opens a file, processes it according to the function, then closes it
\end{itemize}
